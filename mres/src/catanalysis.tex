\chapter{Category Analysis\label{cha:catanalysis}}

This chapter analyses the core categories identified in the corpus, how those categories have emerged from the data and the inter-relationship between them. The identification of categories and the relationships they share is a fundamental process the researcher performs when attempting to generate a theory using Constructivist Grounded Theory (CGT) methodology. Guided by the process of data gathering and analysis, I engaged with secondary literature that contributes to the theoretical density of the emergent categories. With categories, a researcher can elevate any insights they may uncover to a more abstract, theoretical level, whilst at the same time have a grounding in the data. The categories identified in this chapter can help explain how cultural diplomacy was reflected in \textit{Neues Deutschland} articles and the processes the journalists undertook in doing so.

I began the analysis of this thesis by performing an initial line-by-line coding of the articles reporting on the five matches leading up to the FIFA World Cup match between the GDR and FRG. I examined each line to explicate any underlying processes evident within the text. These processes are then coded by using gerunds to ensure that the initial code succinctly describes the action implied by the identified processes. After the first five match reports were analysed, 89 initial line-by-line codes were created. These initial codes capture a wide range of processes, including how journalists ascribed certain attributes to GDRNT players, evaluated performances of GDRNT players and other processes related to the reporting of sports.

By identifying similarities amongst many of the initial codes through the process of simultaneous data analysis and constant comparison of the codes, I was able to create categories that subsumed similar initial codes. This allowed for theoretical engagement of the data at a more abstract, conceptual level. Initial codes were also raised to the status of a standalone category if they could assist in developing an analytic framework for theoretical conceptualisation (\cite{charmaz2014}, p. 189). A total of 41 categories emerged from the cross-comparison and memoing process. Due to the scope of this thesis, a thorough examination of all categories was not possible. I decided to focus on the categories that were relevant to cultural identity and cultural diplomacy. These selected categories form the “core categories” for this thesis’ CGT method. The categories not selected could be used as core categories in future research projects with different research objectives.

The categories selected for analysis all relate to the performance of GDRNT footballers. Each category can be either be classified as ascriptive or evaluative. Ascriptive categories relate to instances when \textit{Neues Deutschland} journalists attributed specific qualities to GDR footballers. The process of attributing specific qualities to GDR footballers enabled journalists to present them as exemplars of the nation’s sporting talents.  Evaluative categories relate to instances when \textit{Neues Deutschland} journalists assessed the performances of GDR footballers according to the desired qualities and performance standards expected from a national team representative. Ascriptive and evaluative categories help illuminate the processes by which \textit{Neues Deutschland} journalists attempted to create a narrative that presented GDR footballers through an ideological perspective that aligned with the goals of the state. These processes enabled the journalists to be critical of GDR footballers when expectations were not met.

\section*{Ascriptive categories}

\subsection*{\textit{Associating Körperkultur with GDR football}}

\textit{Associating Körperkultur with GDR footballers} is a category that has emerged through the process of simultaneous data analysis and constant comparison of the data. Line-by-line coding originally produced four initial codes listed below, along with excerpts from the analysed articles:

\begin{displayquote}
\begin{small}
\textbf{\textit{Associating positive attributes to the GDRNT}}\\
'Unsere Mannschaft begann die Partie wie erwartet: mit hohem Einsatz, tempostark...'\\
(\textit{Our team started the match as expected: with high commitment, fast...})\
\begin{flushright}\footnotesize (\cite{nd19740619})\end{flushright}
\end{small}
\end{displayquote}

\begin{displayquote}
\begin{small}
\textbf{\textit{Characterising a GDRNT player as skilful}}\\
'Wätzlich setzt am linken Flügel Hoffman ein. Dessen präzise Flanke nimmt der Rostocker Streich ungefähr 12 Meter vor dem Tor gekonnt an und hebt sofort das Leder über Clemence ins Netz.'\\
(\textit{Wätzlich starts Hoffman on the left wing. His precise cross is skilfully taken by Rostock’s Streich about 12 metres in front of the goal and immediately lifts the leather over Clemence into the net.})\
\begin{flushright}\footnotesize (\cite{nd19740530})\end{flushright}
\end{small}
\end{displayquote}

\begin{displayquote}
\begin{small}
\textbf{\textit{Highlighting the athleticism of the GDRNT}}\\
'Dann setzte sich die bessere konditionelle und athletische Verfassung der DDR-Mannschaft durch. Das gab den Ausschlag.'\\
(\textit{Then the better conditional and athletic condition of the GDR team prevailed. This tipped the scales.})\
\begin{flushright}\footnotesize (\cite{nd19740615})\end{flushright}
\end{small}
\end{displayquote}

\newpage
\begin{displayquote}
\begin{small}
\textbf{\textit{Stating the effortlessness of the GDRNT in playing at a high tempo}}\\
'Beeindruckend war ohne Zweifel die athletische Bereitschaft der DDR-Nationalmannschaft, die mühelos über die gesamte Spielzeit aufs Tempo drückte.'\\
(\textit{The athletic readiness of the GDR national team was undoubtedly impressive, as they effortlessly kept up the pace throughout the entire match.})\
\begin{flushright}\footnotesize (\cite{nd19740615})\end{flushright}
\end{small}
\end{displayquote}

CGT’s constant comparative method requires the researcher to examine relationships amongst initial codes, potentially identifying any commonalities between the codes. GDR footballers were characterised as fast, athletic and skilful, all of which are attributes desired in a footballer. The four aforementioned initial codes are common in the way they describe the physical ability of GDR footballers. \textit{Associating Körperkultur with GDR footballers} subsumes the four initial codes, synthesising them into a category that examines \textit{Körperkultur’s} usage in East German media.

Known in English as physical culture, \textit{Körperkultur} relates to the activities where the body itself – including its anatomy, physicality and its forms of movement – is the very purpose of the activity (\cite{hargreavesvertinsky2007}, p. 1). Although the characterisation of footballers as fast, athletic and skilful is not unique to the writing of \textit{Neues Deutschland} journalists, its significance is made apparent when viewed through the lens of \textit{Körperkultur} in the GDR. Throughout German history, \textit{Körperkultur} has been closely linked to the goals of the nation and state, shaped by the ideology of those who were in power.

\subsubsection*{\textit{Körperkultur} prior to the GDR}

\textit{Körperkultur’s} existence within the GDR geographic area dates back to the late 18th and early 19th century when pedagogues advocated in favour of the physical education of youth. In 1793, Johann Christoph Friedrich GutsMuths, a teacher at the Schnepfentahl institute, published \textit{Gymnastik für die Jugend} (Gymnastics for the Youth) which became a seminal text for physical education due to how it systematically and scientifically approached the teaching of gymnastics (\cite{naul2002}; \cite{reinhartkrüger2007}, p. 47). Concerned by the Prussian army’s defeat by Napoleonic forces, Friedrich Ludwig Jahn also encouraged the training of young men in gymnastics, helping them to grow stronger and become competent defenders of their homeland (\cite{kaimakamiskirialanisakbanidis2008}, p. 44). Jahn’s advocacy for gymnastics in service to a burgeoning sense of German nationalism led to him being labelled ‘Turnvater’ Jahn, the Father of Gymnastics (\cite{eisenberg1996}). Physical education was introduced into the state school system in the mid-19th century to not only develop the physical capabilities of youth, but to also cultivate their mental faculties (\cite{reinhartkrüger2007}, p. 49). Adolf Spieß, the gymnast and educator, argued that the practice of gymnastics helped the individual to gain complete control of their own body, training the mind to use the body as its most free tool (\cite{reinhartkrüger2007}, p. 50). The result of such training developed disciplined individuals who were more easily integrated into a larger social body.

The era of National Socialism in Germany witnessed the continuation of physical education in servitude to the goals of the state. For Foucault, Hitler’s reign represented the peak of this type of power technology by claiming '… no State could have more disciplinary power than the Nazi regime. Nor was there any other State in which the biological was so tightly, so insistently, regulated' (\cite{foucault1999}, p. 259). True to a totalitarian state, the National Socialist regime exercised control over its subjects’ lives. One form of indoctrination was strategies of persuasion aimed at the body; the regime used physical education as a means of reshaping the subject in a literal sense, promoting stronger and healthier bodies through physical exercise and sports programmes (\cite{keys2009}, p. 395). The National Socialists took up the idea that an individual’s mind could be controlled through the training of the body, developing subjects Adolf Hitler ‘could control as he saw fit’ (\cite{johnsonhumphreyjohnson1957}, p. 12). Physical education under the National Socialist regime was a technology of power\footnote{Michel Foucault (\citeyear{foucault2005}) describes the disciplining of individuals as a technology of power, where the state is able to control the individual bodies of its subjects.} that developed disciplined bodies subjugated by the state.

Due to the National Socialist regime’s racist and anti-Semitic sentiment, \textit{Körperkultur} in National Socialist Germany was racialized. Racial health policies were enacted that were designed to 'cleanse' Germany of individuals who were threats to the biological 'health' of the nation (\cite{burleighwippermann1991}). Racial politics influenced how the National Socialist regime viewed the body (\cite{hirsch2002}). Physical education aimed to not only physically develop the bodies of German youth, but to develop race-consciousness within the minds of German citizens, promoting the 'physical beauty, efficiency and worth' of the German 'race' (\cite{keys2009}). Hitler prioritised the "breeding" of healthy bodies in service to the nation, stating German youth should spend less time on intellectual matters throughout their school day and preferencing more time for physical training ‘covering every type of sport and gymnastics’ (\cite{hitler1971}, pp. 408-409). The inclination toward the physical development of the German "race" would create bodies that were stronger, more disciplined and have a willingness to defend the German nation in future conflicts.

\subsubsection*{\textit{Körperkultur} in the GDR}

Despite the declared rupture with the national-socialist past, the GDR continued the practice of appropriating \textit{Körperkultur} to suit the needs of the state. Although the East German leadership shared the national-socialist belief that \textit{Körperkultur} would help develop citizens capable of defending the homeland, the socialist goals of the state re-conceptualised \textit{Körperkultur} as a means of developing man as the main productive force in society (\cite{sieger1964}, p. 923). However, as the rise of automation in the workplace challenged the need for man to physically develop himself in order to carry out manual tasks, \textit{Körperkultur} need not be solely focused on the physical development of man. Instead, physical development facilitates the development of personality and creativity (\cite{bobnewa1963}, p. 437), the latter of which is what Walter Sieger\footnote{Walter Sieger was an academic based in Leipzig and published his paper in 1964 which positions him to have been active in East German academia. Although more information could not be found on him, his writing suggests that he was an advocate for transforming the conception of man as a productive force, not only in the physical sense, but holistically.} defined as man’s true productive force (1964, p. 931).

Herein lies a fundamental difference in the conceptualisation of \textit{Körperkultur} between the National Socialist regime and East German leadership: whilst National Socialism was more interested in developing physical bodies in servitude to the state, the GDR encouraged its citizens to develop their mental faculties, so long as it aligned with the goals of the state. The use of \textit{Körperkultur} in the GDR illustrates the challenge the East German leadership faced when re-appropriating concepts that were closely associated with the National Socialist regime. In shifting the focus away from race-consciousness and toward the potential benefits of a physically and mentally capable workforce, the East German leadership were able to employ an idea (tainted by a fascist past) to help achieve the state’s socialist goals.

The GDR transformed the traditional function of \textit{Körperkultur} by expanding it beyond the physical education of the nation’s youth. \textit{Körperkultur} in the GDR was associated with an all-encompassing sports movement that was an integral part of East German society. The movement targeted all citizens: schoolchildren and workers alike. The use of gymnastics in school physical education encouraged students to systematically exercise joints and muscles in all conceivable movements and positions (\cite{reinhartkrüger2007}, p. 50). In 1959, First Secretary Walter Ulbricht spread the famous slogan ‘Für jedermann an jedem Ort – jede Woche einmal Sport’ (For everyone in every place – sport once a week), adding in 1968 ‘… - jede Woche mehrmals Sport’ (Sport several times a week) (\cite{bernett1994}, pp. 35-36). Sport was understood as an important part of the socialist way of life through which the population was made, as the motto of the East German Sports Badge claimed, ‘Bereit zur Arbeit und zur Verteidigung der Heimat’ (Ready for work and defending the homeland) (\cite{reinhartkrüger2007}, pp. 53-54). Workers were encouraged to collect sports badges signifying sporting achievements and participate in sport through their \textit{Betriebssportgemeinschaft} (Company sports club) (\cite{grix2008}, p. 408). Physical exercise through the practice of sport allowed one to improve strength, speed, endurance and agility, thereby developing, improving and maintaining movement characteristics required for one’s productive participation in the workforce (\cite{sieger1964}, p. 929).

By \textit{Associating Körperkultur with GDR footballers}, \textit{Neues Deutschland} sports journalists present elite sportspeople as the pinnacle of a physically conditioned GDR citizenry in a self-stylised ‘Workers’ and Peasants’ State’ (\cite{majorosmond2002}, p. 8). As professionalism in sport was prohibited, elite sportspeople were classed as amateurs and could not derive their income from their chosen sport. There was no clear distinction between those who play sport professionally from those who play it leisurely. Viewed from this perspective, elite sportspeople existed on the same “Worker sport” spectrum as non-elite sportspeople, albeit positioned at the spectrum’s extremities. Like the rest of the populace, they were employed in a variety of occupations and often represented their employers in their respective \textit{Betriebssportgemeinschaft}. Being fast, athletic and skilful, GDR footballers were presented as the embodiment of a peak \textit{Körperkultur} desired by the leaders of the East German state to exhibit GDR ascendency in sport and culture.

\subsection*{\textit{Associating militarism with GDR football}}

\textit{Associating militarism with GDR football} emerged as a category after I compared three initial codes and recognised that they all shared the same “militarism” theme. These initial codes documented the use of military language, references to offensive and defensive manoeuvres or a win at all costs attitude in the data.

\begin{displayquote}
\begin{small}
\textbf{\textit{Imbuing militaristic tones to the GDRNT’s performance}}\\
'Unsere Mannschaft mobilisierte die letzten Reserven, um doch noch den Sieg aus dem Feuer zu reißen.'\\
(\textit{Our team mobilized the last reserves to snatch victory from the fire after all.})\
\begin{flushright}\footnotesize (\cite{nd19740619})\end{flushright}
\end{small}
\end{displayquote}

\begin{displayquote}
\begin{small}
\textbf{\textit{Proclaiming the offensive nature of the GDRNT}}\\
'Unsere Mannschaft suchte die Offensive, die Chilenen widmeten ihr Augenmark wiederum mehr der Defensive.'\\
(\textit{Our team was looking for the offensive, the Chileans again devoted their eye marrow more to the defensive.})\
\begin{flushright}\footnotesize (\cite{nd19740619})\end{flushright}
\end{small}
\end{displayquote}

\begin{displayquote}
\begin{small}
\textbf{\textit{Showing the GDR’s disciplined will to win}}\\
'Um so beachtlicher, daß Sparwasser kurz vor dem letzten Pfiff noch einem energischen Versuch unternahm, daraus einen Sieg zu machen…'\\
(\textit{All the more remarkable that Sparwasser made an energetic attempt to turn this into a victory just before the final whistle was blown...})\
\begin{flushright}\footnotesize (\cite{nd19740530})\end{flushright}
\end{small}
\end{displayquote}

\subsubsection*{The GDR: A militarised state}

Situated at the geographic frontline of the Cold War, the GDR state invariably was a militarised state. There was a large focus on human and financial resources in service to the security of the state. Approximately 10 per cent of the working population were involved in the armed forces and other security organs (\cite{dennis2000}, p. 225). Net expenditure of the country’s produced national income spent on Defence and Security totalled 9 per cent in the 1970s and rose to 11 per cent in the 1980s (\cite{diedrichehlertwenzke1998}, pp. 21-24). Along with the \textit{Staatssicherheitsdienst} (State Security Service, commonly known as the Stasi), the \textit{Nationale Volksarmee} (National People’s Army, or NVA) was presented as the ‘ultimate defensive force’, protecting the workers’ and peasants’ states against internal and external enemies (\cite{fullbrook1995}, p. 45). George Orwell once characterised international sport as ‘war minus the shooting’ (\cite{orwell1945}, p. 10), it is no wonder that associations of militarism are implied in the \textit{Neues Deutschland} articles that report on GDR football matches.

\subsubsection*{National Socialism and Prussia}

The association with militarism in East Germany continued on from the National Socialist regime through to the GDR. The GDR emerged from the Soviet-controlled zone of Germany in the immediate post-war period after the defeat of the National Socialist regime. The \textit{Sowjetische Militäradministration in Deutschland} (SMAD) pursued a programme of anti-fascist democratic reconstruction, resuscitating the revival of political life in the Soviet zone under strict directives that were designed to eliminate the socio-economic preconditions of militarism and National Socialism (\cite{dennis2000}, p. 17). To help navigate the contentious issue of restoring an armed force in service to the GDR, the precursor to the NVA, the \textit{Kasernierte Volkspolizei} (People’s Police in Barracks), was tasked with dealing with conventional crime and other specialised tasks like border protection, passport and identity controls and the protection of sensitive buildings (\cite{fullbrook1995}, p. 46). The military defence of the GDR was still largely the responsibility of the 300,000 Soviet forces who were permanently stationed on GDR soil (\cite{fullbrook1995}, p. 10). Although it is clear that militarism continued on from Nationalism Socialism through to the GDR\footnote{Gerhard Sälter (\citeyear{sälter2009}) and Heiner Bröckermann (\citeyear{bröckermann2011}) discuss how militarism was a crucial element in the military history of the GDR, which continued in Soviet Occupied zone after the defeat of the National Socialist regime and continued through to the rule of Honecker. Sälter’s study examines the earlier periods of the GDR’s existence whilst Bröckermann’s focus was largely on the Honecker era.}, its existence was portrayed as justified by its reconceptualisation. Unlike the Third Reich, which harboured ambitions of expansion and domination, militarism in the GDR was associated with the protection of the Workers’ and Peasants’ state in order to achieve the goal of a sustainable socialist democracy.

Before the existence of the Third Reich and the Weimar Republic before that, Prussia was the primary military power in the region of modern-day Germany. The spectre of Prussian militarism has played out contentiously in the histories of the FRG and GDR, with neither state wanting to be associated with Prussian imperialist and expansionist ideas. National Socialist propaganda had used tales of the Prussian King Frederick the Great and Prussian Minister President Otto von Bismarck to glorify German expansion and imperialism (\cite{kroll2001}, pp. 630-631). Before he became GDR Prime Minister, Otto Grotewohl remarked that the FRG ‘in which the old spirit of Bismarck, Wilhelm II and Hitler is kept alive’, will remain as a constant threat of war (\cite{nd19490320}). With overtures such as Grotewohl’s, the FRG was perceived and presented by the East Germans as the inheritor of Prussian and National Socialist imperialism.

The GDR governed territory that once was occupied by the Kingdom of Prussia, thus the opportunity was present to embrace a Prussian military heritage. Prussian soldiers were renowned for their discipline (\cite{jackman2004}). However, due to the Prussia’s historical imperial ambitions and association with the National Socialist regime, claiming this heritage would prove difficult to manage. It was not until the \textit{Preußenwelle}\footnote{\textit{Preußenwelle} refers to the 1970s and 1980s movement within the GDR and FRG, which sought to positively reimagine and reclaim Prussian heritage after it was considered to be an unsavoury era in German history. For more information see Keil (\citeyear{keil2016}); Meyer (\citeyear{meyer2018}); Colla (\citeyear{colla2019}).} (Prussia Wave) hit in the 1970s and 1980s that both sides of Germany were able to publicly reference Prussia positively (\cite{colla2019}, p. 529). East German scholars made heroes out of Prussian historical figures such as Karl August von Hardenberg, Gerhard von Scharnhorst and Friedrich Ludwig Jahn for critiquing the reactionary Junker\footnote{The Junkers were members of landed nobility who owned great estates in Prussia (\cite{taylor2001}, p. 20).} state. The GDR saw the more progressive parts of Prussian history as integral parts of East German historical heritage (\cite{keil2016}, p. 265). This reworked narrative allowed the GDR to navigate an entangled Prussian past, reinforcing the GDR’s status as Germany’s true inheritors of the nation’s progressive past.

With the establishment of the NVA as the GDR’s army, the East German leadership were tasked with a greater share of responsibility for the nation’s defence and security. The NVA dressed their soldiers in a ‘stone grey’ uniform which was reminiscent of the attire worn by Prussian soldiers (\cite{colla2019}, p. 545). The Defence Minister stated that the uniform was an homage to the ‘old German traditions of the People’s Liberation Armies’ in which ‘are embodied the best military traditions of the German people, national dignity and honour’ (Wenzke, cited in \cite{colla2019}, p. 545). The GDR’s highest military decorations were named after the Prussian generals who were involved in the Wars of Liberation: Gerhard von Scharnhorst and Gebhard Leberecht von Blücher (\cite{keil2016}, p. 272).

\subsubsection*{Invoking Prussian discipline}

\begin{displayquote}
\begin{small}
'... der Siegeswille auch dann noch spürbar, als bis in die Schlußphase der Führungstreffer ausblieb.'\\
(\textit{… the will to win could still be felt when the leading goal was not scored until the final phase.})\
\begin{flushright}\footnotesize (\cite{nd19740314})\end{flushright}
\end{small}
\end{displayquote}

\begin{displayquote}
\begin{small}
'Beim Abpfiff des Spiels durch den Italiener Aurelio Angonese hatten sich beide Vertretungen wohl bis zum letzten ausgegeben.'\\
(\textit{At the final whistle of the match by the Italian Aurelio Angonese, both teams had probably spent themselves to the last.})\
\begin{flushright}\footnotesize (\cite{nd19740619})\end{flushright}
\end{small}
\end{displayquote}

Although not referenced directly, the motifs present in the data from the \textit{Associating militarism with GDR football} category draw upon ideas based on the fabled Prussian soldiers renowned for their discipline and performance of duty (\cite{jackman2004}). The match reports contain consistent references to the GDR footballer’s determination and willingness to win.

\subsubsection*{War minus the shooting: A conceptual metaphor}

Sport, like any other practice, has no inherent meaning or values. However, the discourse of sport and the related practices of sport are given meaning and values through their (re)production. When sport is played, it is ascribed meaning to it by those who participate or are audience to its practice (\cite{crawford2004}; \cite{giulianotti2005}). "War minus the shooting" is a conceptual metaphor that embellishes ordinary occurrences in sport as something extraordinary (\cite{lakoffjohnson2003}). Ideas about war are transposed onto sport: engaging with an enemy, being on the offensive or defensive, claiming victories and experiencing losses. Such ideas shift conceptualisations of sport away from its participatory or leisurely characteristics and towards a heightened sense of competitiveness and incidents of incivility (\cite{murray2012}, p. 586).

\subsubsection*{Football matches as "battles"}

\begin{displayquote}
\begin{small}
'In ihrem vorletzten Weltmeisterschafts-Qualifikationsspiel bezwang die DDR-Fußballnationalmannschaft am Mittwochabend vor 95000 Zuschauern den einzigen Rivalen im Kampf um den ersten Platz, Rumänien, verdient mit 2:0.'\\
(\textit{In their penultimate World Cup qualifying match on Wednesday evening in front of 95,000 spectators, the GDR national football team defeated their only rival in the battle for first place, Romania, 2-0.})\
\begin{flushright}\footnotesize (\cite{nd19730927})\end{flushright}
\end{small}
\end{displayquote}

The matches the GDRNT were involved in were typically described as “battles” with a foreign opponent, thus reinforcing the implications of militarism. \textit{Kampf}, the German word for battle, and its adjective forms are used quite consistently in the articles that have been analysed. The characterisation of football matches as "battles" is not unique to the writing of the \textit{Neues Deutschland} journalists. Liz Crolley et al. (\citeyear{crolleyhandjeutter2000}) discuss how media representations of footballers fall largely in line with their stereotypical ‘identities’ based on historical accounts of warfare. The position of the match opponent within NATO or Warsaw Pact alliances does not seem to influence the usage of the term \textit{Kampf} as it is uniformly present in match reports for both factions. This suggests that the characterisation of football matches being “battles” transcends the Cold War context in which the GDRNT played.

\section*{Evaluative categories relating to improvement}

The following two categories relate to the improvement in performances of the GDRNT. For improvement to be recognised, there need to be two points in time in which the GDRNT’s performance was evaluated, with the second point in time being acknowledged as better than the first. These two points in time can be both located within a single match or spread out over multiple matches. Performances would be evaluated according to desired performance standards expected from GDR footballers.

\subsection*{\textit{Highlighting the team’s ability to improve}}

The category \textit{Highlighting the team’s ability to improve} subsumes two initial codes. As the two excerpts below show, the category accounts for the instances where articles either claim the GDRNT’s ability to improve or stating that improvement has occurred. Data related to this category indicate a \textit{positive} outlook held by GDR journalists as it recognises the possibility for an increase in performance from GDR footballers.

\begin{displayquote}
\begin{small}
\textbf{\textit{Claiming the GDRNT’s ability to improve}}\\
'Daß diese DDR-Mannschaft sich allerdings zu steigern versteht, demonstrierte das erste Tor, das der bis dahin schon unermüdlich um geschickte Spielzüge und schnelle Angriffe bemühte Rostocker Joachim Streich erzielte…'\\
(\textit{The first goal scored by Joachim Streich from Rostock, who had already worked tirelessly on skilful moves and fast attacks, showed that this GDR team was capable of improving.})\
\begin{flushright}\footnotesize (\cite{nd19740530})\end{flushright}
\end{small}
\end{displayquote}

\begin{displayquote}
\begin{small}
\textbf{\textit{Stating an improvement in the performance of the GDRNT}}\\
'Wir haben uns gegenüber dem Treffen mit Australien weiter steigern können, doch von unseren zahlreichen Chancen konnte nur eine genutzt werden'\\
(\textit{We have been able to improve on the meeting with Australia, but only one of our numerous opportunities could be taken.})\
\begin{flushright}\footnotesize (\cite{nd19740619})\end{flushright}
\end{small}
\end{displayquote}

\subsection*{\textit{Highlighting the team’s need to improve}}

\textit{Highlighting the team’s need to improve} incorporates two codes that emerged from initial line-by-line coding. As the two excerpts below show, the category accounts for the instances where articles either state there has been a lack of improvement in the performance of the GDRNT or where improvement is clearly required. If the GDRNT has not made any improvement in their performance, then it is implied that the \textit{need} to improve is still present. Data related to this category indicate a \textit{negative} outlook held by GDR journalists as it recognises current performances as below the standard of expectation.

\begin{displayquote}
\begin{small}
\textbf{\textit{Acknowledging a lack of improvement in the GDRNT during the match}}\\
'Verständlich, daß durch die Auswechslungen der Spielfluß in der zweiten Spielhälfte nicht besser wurde.'\\
(\textit{It is understandable that the flow of the match did not improve in the second half of the match due to the substitutions.})\
\begin{flushright}\footnotesize (\cite{nd19740328})\end{flushright}
\end{small}
\end{displayquote}

\begin{displayquote}
\begin{small}
\textbf{\textit{Acknowledging the improvements required for the GDRNT}}\\
'Beim Schlußpfiff des sicher amtierenden Niederländers Keizer waren sich Zuschauer ebenso wie Mannschaft und Trainer einig, daß es in den drei Wochen bis zum ersten WM-Spiel noch allerhand zu tun gibt.'\\
(\textit{At the final whistle of the securely reigning Dutchman Keizer, spectators as well as team and coach agreed that there is still a lot to do in the three weeks until the first World Cup match.})\
\begin{flushright}\footnotesize (\cite{nd19740524})\end{flushright}
\end{small}
\end{displayquote}

\section*{Evaluative categories relating to criticism}

The following four categories relate to criticisms made toward the performances of the GDRNT. Journalists who covered sport were afforded more freedom in how they reported their stories when compared with their counterparts who covered topics such as politics, the economy and international relations. Meyen and Schweiger attribute this freedom to the belief that sport articles hardly contained any politically relevant content (2008, p. 87). Although sport articles may have not referenced political issues directly, it would be difficult to claim that sport articles published in the GDR were completely devoid of politics altogether. The relative freedom experienced by \textit{Neues Deutschland} sports reporters then facilitated their ability to be critical towards representatives of the GDR: the GDRNT footballers. The following categories demonstrate the varied ways in which journalists were able to overtly or covertly criticise GDR footballers. The ability to criticise GDR footballers stems from \textit{Neues Deutschland} journalists evaluating whether GDR footballers were reaching desired performance standards.

\subsection*{\textit{Acknowledging a sub-standard performance from GDR footballers}}

The category \textit{Acknowledging a sub-standard performance from GDR footballers} incorporates seven initial codes that all relate to instances when articles have reported poor on-field performances from GDRNT players.

\begin{displayquote}
\begin{small}
\textbf{\textit{Acknowledging a decrease in the GDRNT performance}}\\
'In der zweiten Hälfte der ersten 45 Minuten verloren unsere Aktionen dann etwas an Präzision.'\\
(\textit{In the second half of the first 45 minutes our actions then lost some of their precision.})\
\begin{flushright}\footnotesize (\cite{nd19740619})\end{flushright}
\end{small}
\end{displayquote}

\begin{displayquote}
\begin{small}
\textbf{\textit{Acknowledging a GDR player’s indiscretion}}\\
'Später wurde auch Gerd Kische verwarnt'\\
(\textit{Later, Gerd Kische was also warned [by the referee].})\
\begin{flushright}\footnotesize (\cite{nd19740619})\end{flushright}
\end{small}
\end{displayquote}

\begin{displayquote}
\begin{small}
\textbf{\textit{Acknowledging deficiencies in the GDRNT}}\\
'Zweifellos offenbarte das Spiel unserer Elf vor allem am Beginn viele Ungenauigkeiten, und Generalproben sollten schon einigermaßen genau verlaufen.'\\
(\textit{There is no doubt that the game of our team revealed many inaccuracies, especially in the beginning, and dress rehearsals should be reasonably accurate.})\
\begin{flushright}\footnotesize (\cite{nd19740530})\end{flushright}
\end{small}
\end{displayquote}

\begin{displayquote}
\begin{small}
\textbf{\textit{Acknowledging the current performance was not as good as previous performances}}\\
'Eine Halbzeit lang erging es unseren Jungen nicht besser, als an den beiden ersten Spieltagen weitaus berühmteren Mannschaften.'\\
(\textit{For one half of the match, our boys did not fare any better than on the first two matchdays of far more famous teams.})\
\begin{flushright}\footnotesize (\cite{nd19740615})\end{flushright}
\end{small}
\end{displayquote}

\begin{displayquote}
\begin{small}
\textbf{\textit{Acknowledging the GDRNT performance was unsatisfactory}}\\
'Bei allem Verständnis dafür, daß unsere WM-Kandidaten meist das Risiko der harten Zweikämpfe scheuten, war ihre Leistung zumindest eine Stunde lang unbefriedigend.'\\
(\textit{With all understanding for the fact that our World Cup candidates usually shied away from the risk of the hard duels, their performance was unsatisfactory for at least one hour.})\
\begin{flushright}\footnotesize (\cite{nd19740524})\end{flushright}
\end{small}
\end{displayquote}

\begin{displayquote}
\begin{small}
\textbf{\textit{Acknowledging the GDRNT’s self-inflicted mistakes}}\\
'Unsere Jungen machten sich zudem eine Halbzeit das Leben selbst ein wenig schwer. Im Bestreben, eine schnelle Vorentscheidung zu erzwingen, waren ihre Aktionen zu überhastet. Es schlichen sich zu viele Abspielfehler ein.'\\
(\textit{Our boys also made life a little difficult for themselves halfway through. In an effort to force a quick preliminary decision, their actions were too hasty. Too many playback errors crept in.})\
\begin{flushright}\footnotesize (\cite{nd19740615})\end{flushright}
\end{small}
\end{displayquote}

\begin{displayquote}
\begin{small}
\textbf{\textit{Negatively comparing the current match with a previous match with the same opposition}}\\
'In einer Begegnung, die nicht das Format des letzten Zusammentreffens beider Mannschaften 1972 in Bratislava erreichte…'\\
(\textit{In a match that did not reach the format of the last meeting of both teams in 1972 in Bratislava…})\
\begin{flushright}\footnotesize (\cite{nd19740615})\end{flushright}
\end{small}
\end{displayquote}

\subsection*{\textit{Avoiding being overtly critical}}

The category \textit{Avoiding being overtly critical} subsumes two initial codes where journalists have been careful to veil their criticisms of GDRNT performances. The two initial codes either record when journalists were torn between being critical and disappointed or when a backhanded compliment was given to the GDRNT and its players, thus concealing the criticism being made.

\begin{displayquote}
\begin{small}
\textbf{\textit{Feeling tension between being disappointed and critical}}\\
'Wir sind sicher nicht ganz zufrieden, ohne an dem bravourösen Kampf unserer Jungen Abstriche machen zu wollen.'\\
(\textit{We are certainly not completely satisfied without wanting to cut back on the brave fight of our boys.})\
\begin{flushright}\footnotesize (\cite{nd19740619})\end{flushright}
\end{small}
\end{displayquote}

\begin{displayquote}
\begin{small}
\textbf{\textit{Giving a backhanded compliment to a GDRNT player}}\\
'Erst als der in der zweiten Halbzeit für Kurbjuweit spielende Wätzlich in der 61. Minute mit einem scharfen Hinterhaltschuß das Signal gegeben hatte, zeigten in der Folge auch Streich, Lauck und Kische, daß sie ihre Schußstiefel doch nicht zu Hause gelassen hatten.'\\
(\textit{Only when Wätzlich, who was playing for Kurbjuweit in the second half, had given the signal in the 61st minute with a sharp ambush shot, Streich, Lauck and Kische showed that they had not left their shooting boots at home after all.})\
\begin{flushright}\footnotesize (\cite{nd19740524})\end{flushright}
\end{small}
\end{displayquote}

\subsection*{\textit{Criticising GDR footballers}}

The category \textit{Criticising GDR footballers} replaces the initial code \textit{Giving a warning to GDR players}. Having only one reference associated to this category, I have decided to elevate the code to the status of category because it records an explicit and overt form of criticism directed to GDRNT players.

\begin{displayquote}
\begin{small}
\textbf{\textit{Giving a backhanded compliment to a GDRNT player}}\\
'Daß Sekunden später die Briten den Ausgleich erzielten, könnte man Jürgen Croy zuschreiben, der von Anfang an nicht sonderlich sicher wirkte, sollte aber allen Spielern als Warnung dienen: Die Minuten nach einem Tor verleiten oft im Übermaß der Freude zum Nachlassen der Konzentration, dies aber nutzen nicht nur englische Fußballer gnadenlos.'\\
(\textit{The fact that seconds later the British equalized could be attributed to Jürgen Croy, who didn't seem very safe from the beginning, but should serve as a warning to all players: The minutes after a goal often tempt in excess of joy to lose concentration, but not only English footballers use this mercilessly.})\
\begin{flushright}\footnotesize (\cite{nd19740530})\end{flushright}
\end{small}
\end{displayquote}

\subsection*{\textit{Deflecting blame for poor performances}}

The category \textit{Deflecting blame for poor performances} subsumes the initial codes \textit{Blaming match conditions for a poor performance} and \textit{Giving reasons for the GDRNT’s poor performances}. This category captures the processes the journalists used when attributing the GDRNT’s sub-standard performances to factors out of their control. In doing so, the journalists absolve GDR footballers from any wrongdoing.

\begin{displayquote}
\begin{small}
\textbf{\textit{Blaming match conditions for a poor performance}}\\
'Auf dem glatten Rasen war es schwierig, vor allem die langen Pässe genau zu addressieren.'\\
(\textit{On the slick lawn it was difficult to precisely address especially the long passes.})\
\begin{flushright}\footnotesize (\cite{nd19740619})\end{flushright}
\end{small}
\end{displayquote}

\begin{displayquote}
\begin{small}
\textbf{\textit{Giving reasons for GDRNT’s poor performances}}\\
'... wobei in Rechnung zu stellen ist, daß offensichtlich am Ende einer kräftezehrenden Saison die Konzentrationsfähigkeit bei einigen Spielern nachläßt...'\\
(\textit{... taking into account the fact that some players seem to lose their concentration at the end of an exhausting season.})\
\begin{flushright}\footnotesize (\cite{nd19740328})\end{flushright}
\end{small}
\end{displayquote}

\section*{Conclusion}

The categories analysed in this chapter will inform the discussion surrounding theory construction in the next chapter. Whilst other categories have not been included in this analysis (which could be potentially useful for another research project), the selected categories are relevant to an examination of how cultural diplomacy was reflected in \textit{Neues Deutschland}. In the CGT methodology, categories are a fundamental tool that allows the researcher to gain theoretical distance from the data, whilst simultaneously being grounded in the data. By identifying two classifications of categories: ascriptive and evaluative, the next chapter will show how the categories relate with one another thereby constructing a theory that explains how cultural diplomacy was reflected in \textit{Neues Deutschland}. 