\chapter*{Introduction}\label{cha:introduction}

This thesis focuses on how the German Democratic Republic’s (GDR, or East Germany) cultural diplomatic efforts through football were reflected in the state-run newspaper outlet \textit{Neues Deutschland}. It analyses the articles that serve as match reports for the GDR national team (GDRNT) performances to discover what cultural values were ascribed to GDRNT footballers. \textit{Neues Deutschland} was selected because foreign statesmen and ambassadors read the content of this publication as official communications from the GDR’s ruling party (\cite{fiedler2014}, pp. 104-109). This, therefore, makes the publication a site in which cultural diplomacy might have been deployed, and there might be read in terms of analysing cultural diplomacy.

As a soft-power approach, cultural diplomacy attempts to influence a foreign audience through the dissemination of culture, values and ideas (\cite{lenczowski2009}, p. 76). The academic literature that surrounds cultural diplomacy tends to focus on its application through the fine arts (\cite{prevots1998}; \cite{david-fox2011}; \cite{mikkonensuutari2016}). Literature on cultural diplomacy’s application through sport is limited. Two possible explanations of why this is the case are offered. First, when one conceives of a cultural system, one would usually associate artefacts such as language, customs, beliefs, knowledge and art as parts thereof (see \cite{boas1930}; \cite{tylor1958}). Although sport could be included as part of the ‘customs’ found within a culture, it has a lesser prevalence in studies relating to culture compared to the aforementioned artefacts. The second explanation is found within the structural framework of higher degree research. The present-day university is typically comprised of several faculties, of which contain many departments and colleges. From a Weberian perspective, present-day universities are exemplary of a rationalised system that privileges efficiency to the detriment of values in the public sphere, which then leads to the fragmentation of all cultural values (\cite{weber1976}). Most academic disciplines have euro-centric origins, filled with their own traditions, theoretical frameworks and methodologies (\cite{klein1990}). Through this rationalisation, some topics of interest may have received more attention from some disciplinary fields than others. For example, the arts have been comprehensively studied by the humanities, whilst sport has captured the interest of the human sciences. The fragmentation of study has led to a lack of collaboration across disciplinary fields and faculties.

There is an identifiable gap in literature on sport’s connection with cultural diplomacy. Therefore, through this thesis, I aim to intervene in cultural diplomatic studies by advocating that sport be recognised as a cultural artefact in the context of cultural diplomacy. Publications on sport’s connection with diplomacy exist, but they have a specific interest in sports diplomacy (\cite{holt1999}; \cite{tomlinson2016}; \cite{cooley2018}; \cite{murray2018}; \cite{rowe2018}; \cite{shuman2018}). However, sports diplomacy is differentiated from cultural diplomacy because the former does not generally focus on the relationship between athletes and cultural values. Sports diplomacy engages with the interactions between states and their agents, using the conduit of sport to foster and maintain diplomatic ties. When conceiving of cultural diplomacy through the arts, there is an explicit emphasis on how art represents the cultural values of a nation (\cite{gonçalves2016}, p. 147; \cite{johnson2016}, p. 14). Following the model of cultural diplomacy through the arts, this thesis will examine how the GDR attempted to embed their cultural values in their athletes. This examination can also uncover further evidence of how the GDR endeavoured to project a cultural identity that was distinct from their western neighbour, the Federal Republic of Germany (FRG, or West Germany).

\newpage
\section*{Literature Review}

To engage with the research question, I have surveyed publications that examine the topics of culture, propaganda and ideology. Additional literature on cultural, public and sport diplomacy more broadly is also reviewed to provide context on the function of cultural diplomacy in relations between and within states. I have also reviewed publications that specifically focus on football in the GDR and the media representation of football to demonstrate how football was used as a cultural diplomatic tool. Combined, these works allow me to position this thesis at the intersection between several debates surrounding sport and cultural diplomacy. By taking an interdisciplinary approach, this thesis argues for the inclusion of sport in research on cultural diplomacy.

\subsection*{Culture and diplomacy}

To help elucidate how cultural diplomacy is understood within this literature review, culture and diplomacy are defined in the following terms. Culture refers to a system of values, customs and ideas shared amongst a group that distinguishes itself from non-members through the use of boundaries (\cite{cohen1985}). Following this definition, nations and states are (large) forms of social organisation in which common cultures can exist. A state does not exist in isolation. Today’s globalised world encompasses numerous states, where diplomacy is fundamental in maintaining stable interstate relations. Paul Sharp (\citeyear{sharp2003}) defines diplomacy as the fostering and maintaining of relations between separate groups, which enables the peaceful co-existence of groups and the avoidance of unnecessary conflict. In this thesis, I refer to states as the main actors in examining diplomacy between groups.

\subsection*{Between cultural, public and sports diplomacy}

In recent decades, cultural diplomacy has become an integral component in a state’s foreign policy arsenal. As opposed to using the ‘stick’ (usually hard-power strategies that include the use of military force or economic sanctions), states can use the ‘carrot’ of cultural diplomacy - a soft-power approach that influences a foreign audience through the dissemination of culture, values and ideas (\cite{snow2008}; \cite{ang2015}; \cite{chitty2016}; \cite{hartig2016}). Like diplomacy in general, one of cultural diplomacy’s primary goals is the avoidance of conflict. However, the discourse on cultural diplomacy has been extended by Ien Ang, Yudhishthir Isar and Phillip Mar who identify two other objectives this practice may achieve:  1) the advancement of national interests through the positive presentation of the nation to the world; and 2) the promotion of a harmonious international order for the benefit of all (\citeyear{ang2015}, p. 370). As such, cultural diplomacy has become an attractive foreign policy alternative for interstate relations due to its non-military approach. Depending on the foreign policy objectives of the state, it can provide opportunities to either project a sense of a state’s cultural superiority over another or foster mutual understanding between states.

Since culture and diplomacy are themselves complex terms, this thesis addresses the lack of clarity in how cultural diplomacy is discussed in scholarly debates. Within areas of international relations scholarship, cultural diplomacy has been conflated with another soft-power approach: public diplomacy (\cite{nye2008}). However, there are some differences between the two types of diplomacy. Whereas cultural diplomacy is a governmental practice that uses official agents and envoys to enact foreign policy objectives through activities such as cultural exchange programmes (\cite{ang2015}, p. 367), public diplomacy relies on the interaction between private groups and interests from one state with another (\cite{cull2008a}, p. 19). While both types of diplomacy attempt to foster mutual understanding between peoples from different states (\cite{hartig2016}, p. 261), public diplomacy does not explicitly outwardly project the cultural superiority of the state. Rather, it seeks to ameliorate the negative external perceptions associated with a state (\cite{melissen2011}, p. 14).

The question then must be asked: if private groups or interests are funded by the state and act in line with the foreign policy objectives of the state, should they be viewed as official agents of the state? With the emergence of non-governmental actors within international relations (e.g. non-governmental organisations, transnational corporations and even international sports federations), the distinction between what is traditional cultural diplomacy and public diplomacy has become unclear. In the context of my research project, the GDR footballers were not official ambassadors or envoys of the state. However, since the GDR state sought to control all aspects of society (\cite{fullbrook1995}; \cite{dennis2000}), categorising the footballers as private groups or interests separate from the state becomes problematic. Though the footballer’s activities can be recognised as occupying a liminal space between cultural diplomacy and public diplomacy, the intent of the state’s diplomacy efforts can help clarify whether these activities were associated with a cultural or public diplomatic approach. Thus, studying how these footballers were represented in media publications can help differentiate cultural diplomacy from public diplomacy.

As this thesis argues that cultural diplomacy can use sport as a vehicle, it is necessary to consider the discourses surrounding sports diplomacy. Today’s globalised world has facilitated the emergence of an extensive and complex international web of sports and sporting organisations which Wolfram Manzenreiter labels the ‘sportscape’ (\citeyear{manzenreiter2008}, p. 414). Much of the existing literature on sports diplomacy focuses on the politics within local and international sporting organisations (\cite{holt1999}; \cite{tomlinson2016}; \cite{cooley2018}), the politicisation of global sporting events such as the FIFA World Cup and Olympics (\cite{xu2008}; \cite{dowse2018}) or the use of sport to foster interstate dialogue (\cite{rowe2018}; \cite{shuman2018}). Stuart Murray’s \textit{Sports Diplomacy} (\citeyear{murray2018}) presents his readers with a comprehensive anthology on sports diplomacy, providing theoretical perspectives of sports role in diplomacy, its history and selected case studies. Much like the aforementioned texts, Murray’s work centres on the political aspects of sports diplomacy. However, there is a brief section that explores how sports diplomacy is used as an expression of a nation’s culture (\cite{murray2018}, pp. 97-102).

Through this interdisciplinary study, I will show that sport can and should be considered as a cultural object within research on cultural diplomacy. Murray’s insight into how a nation’s cultural system could be embedded into sport for diplomatic purposes demonstrates sport’s viability as a medium for cultural diplomacy. It also provides this thesis with a framework on how to engage with the research question. While cultural diplomacy and public diplomacy tend to overlap, cultural diplomacy and sports diplomacy scholarship have been less interchangeable. When conceiving of the parts within the cultural system of a nation or state, many publications point toward the somewhat obvious fields of language, music and art (\cite{hanna1987}; \cite{blackingetal1995}; \cite{clifford1998}). Consequently, publications examining cultural diplomacy exclusively focus on music (\cite{fosler-lussier2015}; \cite{mikkonensuutari2016}; \cite{saito2020}), art (\cite{barnhisel2015}; \cite{mikkonensuutari2016}) and dance (\cite{prevots1998}; \cite{kodat2014}; \cite{mcdaniel2014}), thus omitting sport’s potential to provide a research subject on culture. Literature on sports diplomacy is derived from the international relations scholarship, hence, they have a focus on the political and organisational aspects of sport and neglect its cultural characteristic. By combining perspectives from cultural and sports diplomacy scholarship, research on cultural diplomacy’s application through sport can be achieved.

\subsection*{Cultural diplomacy, propaganda and ideology}

To properly engage with how the GDR used \textit{Neues Deutschland} to reflect their cultural diplomacy efforts in football, we need to consider how cultural diplomacy is differentiated from propaganda and ideology. There may be, both as concepts and practices, some overlap between them. As the primary focus of cultural diplomacy is on influencing a foreign audience through the dissemination of culture, scholars occasionally use the alternate term of ‘cultural propaganda’ to describe the propaganda potential of cultural diplomacy (\cite{prevots1998}; \cite{david-fox2011}; \cite{faircloughwiggins2016}). Making the distinction between the two terms no clearer, Michael David-Fox argues that there has always been a ‘great deal of overlap between propaganda and cultural diplomacy’ (\citeyear{david-fox2011}, p. 15). Though there is debate on whether propaganda is manipulative or neutral (\cite{diggs-brown2011}, p. 48), there is a general consensus in characterising it as an act of persuasion in order to influence people and behaviour (\cite{blackroberts2011}; \cite{auerbachcastronovo2013}; \cite{milleretal2016}). Propaganda’s connection with cultural diplomacy is not accidental. Nicholas Cull recognised the link between the terms when he associated cultural diplomacy’s coinage with the need for a new euphemistic term to describe the soft-power diplomatic programmes of states (\citeyear{cull2008a}, p. 19; \citeyear{cull2008b}, p. 259). While cultural diplomacy may share etymological ties with propaganda, the former is better suited to the focus and scope of this thesis due to its narrower focus on interstate relations.

As the GDR employed a brand of Marxist-Leninist communism to govern its citizens (\cite{grixcooke2002}, p. 17), a discussion of ideology is then essential. Daniele Conversi has defined ideology as ‘a set of ideas articulated around a socio-political programme’ (\citeyear{conversi2010}, p. 26). In the GDR context, one could recognise Marxism-Leninism as the ideological model that directed the nation (\cite{grixcooke2002}, p. 17). Mary Fullbrook maintained that the East German state was not clearly separate from society as ‘there was to be \textit{no} area of society uncontrolled by the state’ (\citeyear{fullbrook1995}, p. 19, emphasis in original). Shifting the perspective slightly, Anselma Gallinat reformulated Fullbrook’s claim in suggesting there was no separation between official ideology and grassroots practice (\citeyear{gallinat2005}, p. 291). The combination of the two perspectives allows us to view the state and ideology as one in the same. If there was no separation between state and society, ideology and the grassroots, then the cultural values associated with the state’s cultural diplomacy efforts would be the state-endorsed ideology. Cultural diplomacy’s intended target is foreign audiences. However, when the GDR citizens viewed the cultural diplomatic efforts of the state through the reading of \textit{Neues Deutschland}, then it could be seen as the proliferation of ideology and the ‘sustaining of relations of dominance’ (\cite{thompson1990}, p. 58). While cultural diplomacy may not be intended for consumption within the boundaries of the instigating state, the GDR’s context transforms cultural diplomatic efforts into the proliferation of ideology when their citizens are exposed to its dissemination.

\subsection*{Cultural diplomacy through the arts}

Most literature on cultural diplomacy focuses on the arts and high culture. For this reason, the insights they offer are valuable for a point of reference for sport’s role in cultural diplomacy given that both the fine arts and sport are parts of the cultural system. The focus of the existing literature on the arts and high culture means that it does not provide nuanced insight into mass culture, and by extension, sport. Moreover, literature on cultural diplomacy through the fine arts typically uses the context of Soviet-American relations in the 20th-century as a background for analysis. Thus, there are limits to its applicability in this thesis given its specific focus on the relations between two distinct states who have a shared German heritage.

David-Fox’s \textit{Showcasing the Great Experiment} (\citeyear{david-fox2011}) provides a historical account of how the Soviet Union used the ‘All-Union Society for Cultural Ties Abroad’ (better known by its Russian acronym “VOKS”) to showcase Soviet artistic excellence to a foreign audience. By examining Soviet archives from the interwar period, David-Fox offers an insight into the Soviet state’s perspective on cultural diplomacy and the role it played in projecting the perception of socialism’s superiority. Of particular relevance to this thesis is David-Fox’s understanding of how the Soviet Union attempted to improve its international reputation despite being considered culturally ‘backward’ and economically weak (\citeyear{david-fox2011}, p. 9).

\textit{Music, Art and Diplomacy}, edited by Simo Mikkonen and Pekka Suutari (\citeyear{mikkonensuutari2016}), offers critical perspectives on cultural diplomacy from both sides of the Iron Curtain. Focusing on the 1940s to the 1960s, the volume’s chapters detail specific acts of cultural diplomacy, where states sent artists to their ideological enemy to showcase their talents. Rather than investigating the state’s perception of cultural diplomacy, \textit{Music, Art and Diplomacy} focuses on the perspectives of the artists and audience themselves, offering an alternative to the grand narratives that usually dominate East-West discourse in Cold War research. The volume utilises archival sources alongside oral history through personal testimony to construct narratives of personal experiences in cultural diplomacy. Thus, the text constitutes a valuable resource for this thesis – it recounts how the targets of cultural diplomacy received the artists’ performances.

Naima Prevots’ \textit{Dance for Export} (\citeyear{prevots1998}) references sources from official American government archives to present an entirely American viewpoint on the value of cultural diplomacy during the Cold War. \textit{Dance for Export} chronicles how the United States government channelled their cultural diplomacy efforts through the American National Theatre and Academy (ANTA) by sending dancers abroad to exhibit excellence in Western ballet, thus counteracting their perceptions as an ‘uncultured, superficial, and materialistic’ nation (\cite{prevots1998}, p. 7). Though these texts are specific to art and high culture, they serve as a very useful comparative study for examining how cultural diplomacy was implemented through the mass medium of sport.

The outlined approaches examine how cultural values were embedded in the artistic practices utilised in the cultural diplomatic programmes employed by states. By drawing on these approaches, I will construct an analogous theory of cultural diplomacy’s use in sport. In providing historical accounts of cultural diplomacy’s use in the 20th-century, these texts cover a variety of perspectives and time periods, which together present a well-rounded understanding of cultural diplomacy. The texts all agree that state actors used cultural diplomacy to promote their values to their ideological opponents while projecting a sense of cultural superiority. However, only David-Fox is explicit in acknowledging cultural diplomacy’s effect on domestic populations (\citeyear{david-fox2011}, p. 122). This insight suggests that cultural diplomacy had a multi-directional effect: it influenced audiences inside and outside the territorial bounds of the state. To explain why the fine arts were the preferred vehicle for cultural diplomacy, \textit{Dance for Export} and \textit{Music, Art and Diplomacy} both argue the fine arts could transcend linguistic barriers and political ideologies, therefore increasing its effectiveness in influencing foreign audiences (\cite{prevots1998}, p. 19; \cite{gonçalves2016}, p. 142). Additionally, Mikkonen and Suutari claim that cultural diplomacy is more effective than traditional diplomacy because of its ability to appeal to emotions (\citeyear{mikkonensuutari2016}, p. 4). Since the fine arts are often associated with high culture, all three texts subsequently have an implicit focus on the privileged classes of society. Sport is not typically associated with high culture and enjoys support from a wide cross-section of society. Like the fine arts, sport can elicit emotional responses from spectators, and as such, studying football allows for an exploration of cultural diplomacy in relation to mass culture.

\subsection*{Football and \textit{Neues Deutschland}}

The sport of football provides a research subject rich with cultural attachments, with media coverage of football allowing for an examination into cultural diplomacy. Viewed through the lens of Anthony Cohen’s work on symbolism (\cite{cohen1985}, p. 17), football provided the GDR with a signifier that allowed it to associate the state’s cultural values with their national team players. Only \textit{Reisekader} (people who were loyal to the regime) were allowed to travel abroad to watch the national team play and spectators for international games in the GDR were heavily vetted before they were allocated a ticket (\cite{mcdougall2014}, p. 178). According to Helmut Hanke, everyday cultural life in the GDR was dominated by devotion to the house and the impact of the media (\citeyear{hanke1990}, p. 179). Therefore, articles published in \textit{Neues Deutschland} would have been one of the primary ways in which people engaged with football.

Literature that examines GDR football’s relationship with East German culture is limited. Alan McDougall’s \textit{The People’s Game} (\citeyear{mcdougall2014}) is the most comprehensive English-language text on football in the GDR. While it offers a thorough historical account of GDR football, McDougall does not discuss GDR culture or media representation in great detail. A collection of literature exists on football’s association with cultural identity (\cite{archetti1994}; \cite{gibbons2014}; \cite{bocketti2016}) and football’s representation in the media (\cite{boylehaynes2004}, \cite{bridgewater2010}; \cite{miller2011}). Publications that interest themselves with national identity have relevance to this thesis in the way they analyse how national culture can be embedded in football. There are however limits to their applicability to this thesis as the texts do not explicitly question how the cultures of separate nations interact with one another.

Literature that focuses on football’s media representation gives an insight into how media organisations construct narratives around the sport. These texts are limited in their suitability to this thesis as they lack analyses of cultural representations produced in media publications and instead primarily focus on the economics of football and the sport’s growing ‘brand’ (\cite{bridgewater2010}). Furthermore, I have yet to find any English-language publications examining the influence that \textit{Neues Deutschland} had within the GDR. Although I have achieved German-language proficiency at the C1 level (CEFR standard) through my undergraduate education in German language and culture, this thesis is positioned as an English-language work. Due to the dearth of English-language literature on GDR football and \textit{Neues Deutschland}, this thesis utilises publications that specifically engage with the research topic in the German language as an exception.

\textit{Denver Clan und Neues Deutschland} (\cite{meyen2003}) and \textit{Sattsam bekannte Uniformität} (\cite{meyenschweiger2008}) are two publications that go beyond a rudimentary examination of \textit{Neues Deutschland’s} influence in the GDR and instead demonstrate how content analysis of articles can be performed. While the two texts could suffer from post-unification distortion (\cite{fullbrook1995}, p. 7), they apply to this thesis in the way they can guide the methodology. They provide a framework that demonstrates how to engage with the text written in \textit{Neues Deutschland} whilst understanding the context from which the articles came. As the majority of people in the GDR who engaged with football did so through the consumption of media, the articles published by \textit{Neues Deutschland} provide this thesis with a primary source of how football was represented in the context of the research topic.

\section*{Scope}

The thesis analyses text in \textit{Neues Deutschland} articles relevant to the GDRNT between October 1972 and June 1974 to uncover how football was used as a tool of cultural diplomacy. Since the advent of international competition, sport has been used as a political tool by competing nation-states. This is particularly evident in the sport of football, where the nation seems more tangible ‘as a team of eleven named players’ (\cite{hobsawm2012}, p. 143), becoming a symbol of the ‘imagined community’ (\cite{anderson1983}) of the state. Since the GDR began to compete in international competitions as an independent nation, it was referred to as the East German ‘miracle’ due to their achievements in international sports competitions. The GDR enjoyed great success in Summer and Winter Olympic games throughout their existence despite their relatively small population (\cite{dennisgrix2012}). The GDR attributed its sporting success to the ‘good socialist’ character of their athletes. (\cite{dennisgrix2012}, p. 21). In contrast, the GDRNT experienced mediocre results in European competition and qualified just once for the FIFA World Cup in 1974, 22 years after the team’s first international match.

Considering GDR footballers were not as successful as their Olympic counterparts, studying football allows for an interesting examination into how football was (re)presented in the state-run newspaper. As such, instead of focusing on sport more broadly, I have chosen to narrow my research scope to football because of the complementary narrative it provides against the assertions of the East German dominance in international sport. The FIFA World Cup the GDR qualified for was hosted in the FRG. Adding to the significance of the event, the GDR and FRG were drawn in the same competition group. The two Germanys faced each other on the football pitch for the first (and only) time. The context of this match allows for an investigation of cultural diplomacy between two states who are ideologically differentiated yet have a shared national heritage.

This study analyses newspaper articles that report on the GDRNT’s performances. Though other newspapers existed in East Germany, such as \textit{Junge Welt} \textit{Berliner Zeitung} and \textit{Neue Zeit}, I have chosen to examine newspaper articles from \textit{Neues Deutschland} for three main reasons. First, \textit{Neues Deutschland} was the official newspaper of the \textit{Sozialistische Einheitspartei Deutschlands} (Socialist Unity Party of Germany, or SED). Typical of a communist one-party state, the SED governed East Germany throughout the state’s existence. This made the newspaper the mouthpiece of the government. Analysis of the articles will help uncover the state’s attempts of cultural diplomacy found in the representations of the GDRNT and its players. Secondly, \textit{Neues Deutschland} was the most circulated newspaper in the country with the largest readership (\cite{meyenschweiger2008}). As such, the newspaper would have been an effective way to communicate with GDR citizens and foreign audiences. Finally, \textit{Neues Deutschland} has digitised and archived its entire publication history on the website www.nd-archiv.de. With an archive subscription, one could use the websites search function to find any article ranging from the first publication on the 23rd of April 1946 to the 3rd of October 1990, when Germany officially reunified. Alongside plain text results that show the content of the articles, high-quality image scans in the form of PDF files are also available.

I have chosen to examine articles ranging from the GDRNT’s first World Cup qualifying match up until the penultimate match before their clash with the FRG. Due to the contrasting results experienced by both teams in previous international competitions, analysing articles published in the months before the game will help illuminate how the East German state attempted to use cultural diplomacy in the face of adversity. As East Germany had eventually won the match against West Germany, articles published after the match will have little relevance to my research as it is more interesting to see how the GDR portrayed lead up to the match. Consequently, I have collected articles from the dates between the 8th of October 1972  (the commencement of the GDR’s World Cup qualifying campaign) and the 22nd of June 1974 (the day before the GDR and FRG match).

\section*{Method}

To discover processes in \textit{Neues Deutschland} articles that reflect the state’s attempt of cultural diplomacy through football, I employed a method that is guided by Constructivist Grounded Theory (CGT) methodology. Grounded theory is a qualitative methodology that involves systematically gathering and analysing data to construct theory. It is typically used in sociological research that involves analysing interview data (\cite{oktay2012}; \cite{bryantcharmaz2007}; \cite{bryantcharmaz2019}). Studies using extant texts rather than interviews are sporadic,\footnote{Whilst Grounded Theory studies using extant texts do exist, they usually involve analysing interviews and biographies (\cite{plummer1983}; \cite{yarwood-rossjack2015}; \cite{ravenek2017}).}  but they demonstrate the methodology can still discover processes in non-interview-based research.\footnote{Helen Hardman’s (\citeyear{hardman2013}) publication examining democratisation is a rare example of research project utilising non-interview or biographical extant text as the primary source of data in a Grounded Theory study. Hardman was able to discover the processes leaders used in Soviet states to broadcast a common message to their citizens.}  Originally developed by Glaser and Strauss (\citeyear{glaserstrauss1977}), grounded theory requires the researcher to use inductive reasoning to recursively interpret meaning from the analysis of data. Grounded Theory is both the result of the research and the methodology that guides it. When a Grounded Theory is realised, it offers a theory grounded in the data explaining ‘what is going on’ (\cite{glaserstrauss1977}, p. 23) in the area studied. The researcher can use their findings to subsequently perform a more elaborate process of data gathering and analysis. They can repeat this cyclical process until they are satisfied with how the constructed theory correlates with the data (\cite{glaserstrauss1977}). Charmaz’s CGT builds on the original grounded theory method through the paradigmatic views found in social constructivism.

According to Charmaz (\citeyear{charmaz2014}), CGT is ontologically different from the original grounded theory because it rejects the existence of an objective reality and instead accepts the existence of multiple realities. CGT acknowledges the role of the researcher by ‘acknowledging the interactive and interpretive nature of data construction’ (\cite{inabakakai2019}, p. 336) enabling them to be reflexive in their research. The ability to be reflexive allows this research to respectfully situate itself in studies of the GDR, an area of research often experiences post-unification distortion (\cite{fullbrook1995}, p. 7). As a researcher who is based in Australia with no historical links to the GDR, I am aware that my perspectives of the GDR may be influenced by my educational background and experiences. Thus, being guided by the data when using the CGT methodology allows me to research the GDR with respect and consideration.

I started to collect \textit{Neues Deutschland} articles prior to the official commencement of this Masters’ study. \textit{Neues Deutschland} offers paid archival access to their newspapers from before German reunification. Initially, the website’s search feature was used to find articles that reported on the GDRNT. This method proved to be unreliable as sections of \textit{Neues Deutschland} papers were not recognised by the website’s Optical Character Recognition software. This meant that some articles that reported on the GDRNT were not returned in the search query results. This changed how articles were sourced from the website. Guided by the scope of the research, every daily edition of \textit{Neues Deutschland} was examined for articles on the GDRNT, from the 8th of October 1972 to the 19th of June 1974. A total of 122 articles were transcribed and saved as Text files on the computer. PDF copies of the corresponding \textit{Neues Deutschland} pages were also saved for posterity.

Having performed a close reading on the articles when transcribing and saving them, it became evident that focusing on the match reports of GDRNT would be the ideal place to start data analysis. The match reports are suitable for examining the cultural diplomatic efforts of the state as they contained writing which described and recounted the performances of the GDRNT. The decision to work backwards from the last match report transcribed derived from the belief that articles reporting on the World Cup Finals matches would elicit richer analysis into cultural diplomacy than earlier articles due to importance of the matches. CGT methodology requires the researcher to code by using gerunds\footnote{Gerunds are non-finite verb forms that describe action. Gerunds are words that typical end with the suffix -ing.} to capture the actions in the data (\cite{charmaz2014}, pp. 120-121). The focus on capturing actions in the data allows the researcher to uncover underlying processes present in their research subject. NVivo 12 software was used to manage the coding of articles, allowing for an effective cross comparative process when dealing with a large data set and multiple codes.

Line-by-line coding was performed on the articles reporting on the World Cup Finals matches against Chile\footnote{Appendix B contains an example of the NVivo 12 line-by-line coding of the WCF match against Chile.} and Australia and the three final World Cup preparation matches against England, Norway and Czechoslovakia. Through the process of constant comparison between initial codes and the data, insights emerged from the analysis and were recorded in research memos (\cite{charmaz2014}). In GCT methodology, a researcher uses memos to guide and shape their analytical insights (\cite{charmaz2014}). Memos also help the researcher order analytical insights into an emergent theory (\cite{charmaz2014}). The initial line-by-line coding of the five articles produced categories that subsumed the initial codes or transformed initial codes into standalone categories if they displayed theoretical potential. All subsequent articles were coded with the categories.\footnote{Appendix C contains an example of the NVivo 12 category coding of the 1-0 friendly win against Belgium.} If any coded references did not relate to any existing categories, new initial codes would be created and subsequently examined whether there were new emergent categories. The cross-comparative process of analysing and gathering data, whilst reflecting on any emerging theoretical insights requires the researcher to engage with additional secondary literature to assist in theory construction. Engaging with secondary literature in the memoing process obliges the research to perform an additional literature review that provides contextual understanding of the theoretical categories. Chapter 2 discusses the emergence of the categories from the initial codes in more detail with context provided by the complimentary secondary literature.

Theoretical saturation was achieved well before analysing all 23 match reports within the scope of the thesis. In CGT methodology, theoretical saturation marks an important stage of the research process as it refers to the point when no new processes are found in the data (\cite{charmaz2014}, p. 345). No new initial codes or categories emerged from the cross-comparative process after the 11th article was analysed. All existent categories were able to capture the coded references thereafter. The implication of reaching theoretical saturation in the context of this study suggests that the CGT method captured all the processes relating to cultural diplomacy evident in the match reports of the GDRNT performances. Chapters 3 and 4 will discuss how these processes could be raised to an abstract level and generate a theory that can explain how cultural diplomacy was reflected in \textit{Neues Deutschland}. 
