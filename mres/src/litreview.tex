\chapter{Literature review\label{cha:litreivew}}

\section{Culture and diplomacy}

As my thesis focuses on cultural diplomacy, specifically the use of sport as a form of cultural diplomacy, it would be constructive to examine how the topic has been used and discussed in the academe. A semantic deconstruction of the term provides us with two distinct parts: culture and diplomacy. 

\subsection{Culture}

Culture is a term that has been heavily contested in scholarly debates emanating from various disciplines. The 19th-century British anthropologist Edward Tylor defined culture as ‘that complex whole which includes knowledge, belief, art, law, morals, customs, and any other capabilities and habits acquired by man as a member of society’ (\cite{tylor1958}, p. 1). Tylor’s definition implies that culture is permanent and static because it is \emph{learned} and \emph{acquired}. It exemplifies an earlier, more rudimentary understanding of culture; that is, the aspects of humankind which are not biological. Kroeber and Kluckhohn argued that Tylor’s definition of culture was too broad and included too many distinct components to be a coherent working description (\cite{kuper1999}, p. 57). Franz Boas, who is widely regarded as the founder of Anthropology (\cite{stocking1960}), spent decades formulating an alternative definition of culture to the one proposed by Tylor:

\begin{quote}
    Culture embraces all the manifestations of social habits of a community, the reactions of the individuals as affected by the habits of the group in which he lives, and the products of human activities as determined by those habits. 
    
    \begin{flushright}
    (\cite{boas1930}, p. 79)
    \end{flushright}
\end{quote}

Boas’ conception of culture is differentiated from Tylor’s as it proposes that it is dynamic. Not only can culture influence an individual, but the individual can also influence culture.

Whilst Boas’ definition would heavily influence the field of anthropology throughout most of the 20th-century, the post-modern turn of the 1950s and 60s brought with it new challenges to widely accepted understandings of culture. Whilst it was common to attribute the sharing of a common culture as a predetermined characteristic of a cultural group, Barth proposed the view that the sharing of a common culture was a result of social group organisation (\cite{barth1969}, p. 11). Anthony Cohen advanced the idea that social groups were not strictly defined by a shared system of values or ideas, but rather the boundaries that the group recognised which determined group membership (\cite{cohen1995}). This idea places importance on the social aspect of culture and provides an explanation of how cultural groups can change their shared values throughout time, whilst maintaining a distinct cultural identity. More recently, Abu-Lughod has explored the subject of “halfies” (people whose national or cultural identity is mixed) to claim that culture ‘operates in anthropological discourse to enforce separations’ amongst groups of individuals (\cite{abu-lughod2008}, p. 62). Although these anthropological conceptions of culture have some variations amongst them, what they all have in common is the idea of social organisation amongst individuals who have a sense of shared identity, customs, values and ideas.

Berger and Luckmann’s seminal text The Social Construction of Reality (1966) offers a sociological explanation for the emergence of cultural systems. Built on the premise that humans seek to create order in a chaotic world because of their lack of instinctual equipment (\cite{gehlen1980}), Berger and Luckmann claim that humans, through social interactions form habits out of repeated actions to navigate the world around them (\cite{bergerluckmann1966}). Over time, these habits become traditions and then are institutionalised as abstract objects themselves (\cite{bergerluckmann1966}, pp. 65-109). From the sociological perspective, these traditions and institutions (such as language, rituals and customs) become parts of a larger whole: the cultural system. This perspective allows one to both recognise cultural systems as ‘real’ and ‘constructed’ simultaneously. They are not permanent or unchanging, but only appear to be in certain cases since change within a cultural system happens over an undefined period of time. Much like Berger and Luckmann, Zygmunt Bauman argues that cultural systems emerge from human beings wanting to create some order amongst the \emph{Weltoffenheit} that they are presented with. The paradox that humans face in wanting to create order whilst wishing to determine one’s own future, according to Bauman, is the very mechanism by which cultural systems manifest and evolve (1995, p. 141). Bauman credits the advent of modernity for the manufacture of high and mass culture, where the elite class could choose their form of consumption, thereby producing high culture. Due to the social status of the elite, high culture became the hegemonic culture for the population. The masses, who could not choose for themselves, had to be content with the culture given to them by the elite through a top-down approach (\cite{bauman1995}, p. 146). Bauman’s work on cultural systems gives his readers a convincing account of how a top-down transmission of culture operates within a large population of individuals, namely states and nations.

The brief survey on the different perspectives of culture both within the anthropological and sociological disciplines reveals that the understanding of culture - much like the concept itself - has evolved over the past two centuries. For the purposes of this thesis, culture will be conceived of as a system of values, customs and ideas shared amongst a group that distinguishes itself from non-members through the use of boundaries. Culture is both something that exists and is produced, contingent on the composition of the group’s members.

\subsection{Diplomacy}

Like culture, diplomacy has been ‘a notoriously tricky term… conveying many and different things’ (\cite{sharp2003}, p. 857) in scholarly discussion. Within the confines of International Relations, Berridge et al. define diplomacy as ‘the official channels of communication employed by the members of a system of states’ in order to ‘enable states to secure the objectives of their foreign policies without resort to force’ (\citeyear{berridge2001}, p. 1). Broadening the conception of diplomacy beyond interstate relations, Sharp understands diplomacy as the 

\begin{quote}
    way in which relations between groups that regard themselves as separate ought to be conducted if the principle of living in groups is to be retained as good, and if unnecessary and unwanted conflict is to have a chance of being avoided. 
    
    \begin{flushright}
    (\cite{sharp2003}, p. 858)
    \end{flushright}
\end{quote}

Whether referring to states or groups, diplomacy has historically involved the interaction between two (or more) disparate groups in order to negotiate one’s position, foster mutual understanding or maintain an open line of communication. Hamilton and Langhorne theorise of a time when diplomacy, as opposed to combat, was used to mediate tensions over resource and territory between rival pre-historic human groups (\citeyear{hamiltonlanghorne2011}, p. 148). The advent of fiefdoms, kingdoms and states required a more formal approach to diplomacy between interested parties. During antiquity, it was common practice for the palace bureaucracy to dispatch envoys that represented the authority of the sovereign, to foreign courts or embassies on diplomatic missions (\cite{cohen2013}, p. 16). For the Byzantine Empire, who were surrounded by ‘barbarians’ to the north and west, diplomacy became central to their survival. Unlike their Roman predecessor, the Byzantines could not simply crush their adversaries on the battlefield and instead relied on their diplomatic missions to court the rulers of various tribes, nations and empires (\cite{jönssonhall2005}, pp. 47-49). In the modern post-Westphalian system of nation-states, diplomacy has customarily become the remit of the Ministry of Foreign Affairs, who act in accordance to the foreign policy objectives of the state (\cite{langhorne2000}, p. 33). Regardless of the era in which diplomacy is practised, its function has largely remained unchanged.

If we are to accept the premise that one of diplomacy’s primary objectives is the avoidance of war, then it could be, as Murray claims, a civilising institution (\citeyear{murray2018}, p. 8). Cohen argues that diplomacy has been fundamental to the stable relations between societies across all periods of history (\citeyear{cohen2013}, p. 16). With this stability, humankind has advanced through to more complex forms of social organisation. The civilising nature of diplomacy also lends itself to be closely associated with culture. The sharing of ideas and knowledge between disparate groups of people is made far easier when diplomatic ties are created and maintained. As such, studying cultural diplomacy allows researchers to deep dive into the civil connections between groups locally and on a wider global scale.

\section{Between cultural, public and sports diplomacy}

In recent decades, cultural diplomacy has become an integral component in a state’s foreign policy arsenal. As opposed to using the ‘stick’ (usually hard-power strategies that include the use of military force or economic sanctions), states can use the ‘carrot’ of cultural diplomacy, a soft-power approach that influences a foreign audience through the dissemination of culture, values and ideas (\cite{lenczowski2009}, p. 76). Many scholars attribute cultural diplomacy’s increased usage due to the proliferation of transnational connections in today’s globalised world (\cite{snow2008}; \cite{ang2015}; \cite{hartig2016}; \cite{chitty2016}). While it can be said that one of cultural diplomacy’s primary goals is the avoidance of war (as is the general goal of diplomacy), Ang et al. identify two other objectives that cultural diplomacy may achieve:  the first is ‘advancing the national interest by presenting the nation in the best possible light to the rest of the world’ and the other is ‘to promote a more harmonious international order to the benefit of all’ (\citeyear{ang2015}, p. 370). As such, cultural diplomacy has become an attractive foreign policy alternative for interstate relations due its non-military approach and, depending on the foreign policy objectives of the state, provides opportunities to either project a sense of a state’s cultural superiority over another, or foster mutual understanding between states.

Within the discipline of International Relations, cultural diplomacy has been conflated with another soft-power approach: public diplomacy. However, there are some distinctions between the two types of diplomacy. Whereas cultural diplomacy is a governmental practice that employs the use of official agents and envoys to enact foreign policy objectives through activities such as cultural exchange programs (\cite{ang2015}, p. 367), public diplomacy relies on the interaction between private groups and interests from one state with another (\cite{cull2008}, p. 19). Whilst both types of diplomacy attempt to foster mutual understanding between peoples from different states (\cite{hartig2016}, p. 261), public diplomacy does not explicitly outwardly project the cultural superiority of the state. Rather, it seeks to ameliorate the negative external perceptions associated with a state (\cite{melissen2011}, p. 14). The question then must be asked: if private groups or interests are funded by the state and act in line with the foreign policy objectives of the state, can they then be conceived of as official agents of the state? With the emergence of non-governmental actors (Non-governmental organisations, transnational corporations and even international sports federations) within international relations, it appears that the distinction between what is traditional cultural diplomacy and public diplomacy has become unclear.

As this thesis argues that cultural diplomacy can use sport as its vehicle, we will need to consider what discourse surrounds the topic of sports diplomacy. Today’s globalised world has facilitated the emergence of an extensive and complex international web of sports and sporting organisations which Manzenreiter labels the ‘sportscape’ (\citeyear{manzenreiter2008}, p. 414). Much of the literature that exists on sports diplomacy tends to focus on the politics within local and international sporting organisations (\cite{holt1999}; \cite{tomlinson2016}; \cite{cooley2018}), the politicisation of global sporting events such as the FIFA World Cup and Olympics (\cite{xu2008}; \cite{dowse2018}) or the use of sport to foster interstate dialogue (\cite{rowe2018}, \cite{shuman2018}). Stuart Murray’s Sports Diplomacy (2018) present his readers with a comprehensive anthology on the topic of sports diplomacy, providing theoretical perspectives of sports role in diplomacy, its history and selected case studies. Much like the aforementioned texts, Murray’s offering tends to focus on the political aspects of sports diplomacy. However, there is a brief section that explores how sports diplomacy can be used as an expression of a nation’s culture (\cite{murray2018}, pp. 97-102). Murray’s small insight into how a nation’s cultural system could be embedded into sport for diplomatic purposes opens the door for further research, or more specifically, this thesis paper. Whilst cultural diplomacy and public diplomacy tend to mix in with each other with quite some ease, it appears that cultural diplomacy and sports diplomacy have so far acted like water and oil. To offer a rudimentary explanation for this I could point toward the lack of collaboration between traditional - and often times - rigid disciplinary fields in the academe. Another potential reason for the lack of research on sport as a form of cultural diplomacy could be found in the biases some academics may hold toward the study of sport. When conceiving of the parts within the cultural system of a nation or state, many texts point toward the somewhat obvious fields of language, music and art. I would argue that academics within the humanities have a tendency to overlook sport, whether consciously or unconsciously. As such, I intend to move the needle ever so slightly in favour for the study of sport as a cultural object.

\section{Cultural diplomacy through the arts}

As most literature of cultural diplomacy focuses on the fine arts, the insights they offer will be valuable for a comparable study of sport’s role in cultural diplomacy. Literature on cultural diplomacy through the fine arts typically uses the context of Soviet-American relations in the 20th-century as a background for analysis. 

David-Fox’s Showcasing the Great Experiment provides a historical account of how the Soviet Union used the ‘All-Union Society for Cultural Ties Abroad’ (better known by its Russian acronym “VOKS”) to showcase Soviet artistic excellence to a foreign audience (\cite{david-fox2012}). By examining Soviet archives from the interwar period, David-Fox offers an insight into the Soviet state’s perspective on cultural diplomacy and the role it played in projecting the perception of socialism’s superiority. Of particular relevance to this thesis is David-Fox’s insight into how the Soviet Union attempted to improve their international reputation despite being considered culturally ‘backward’ and economically weak (\cite{david-fox2012}, p. 9). Edited by Mikkonen and Suutari, Music, Art and Diplomacy is a volume of assorted chapters offering perspectives on cultural diplomacy from both sides of the Iron Curtain (\cite{mikkonensuutari2016}). Focusing on the period between the 1940s and 1960s, the volume’s chapters detail a specific act of cultural diplomacy, where states sent artists to their ideological enemy to showcase their talents. Rather than investigating the state’s perception of cultural diplomacy, Music, Art and Diplomacy interests itself with the perspective from the artists themselves, which offers an alternative to the grand narrative’s that usually dominate East-West discourse in Cold War research. The volume utilises archival sources alongside oral history through personal testimony to construct narratives on personal experiences in cultural diplomacy.  Prevot’s Dance for Export references sources from official American government archives to show an entirely American viewpoint on cultural diplomacy’s value in the Cold War (\cite{prevots2012}). Dance for Export chronicles how the United States government channelled their cultural diplomacy efforts through the American National Theatre and Academy (ANTA), by sending dancers abroad to exhibit excellence in Western ballet to counteract the perceptions of an ‘uncultured, superficial, and materialistic’ nation (\cite{prevots2012}, p. 7). Though these texts are specific to art and high culture, they are relevant because they provide me with a comparative case study of how cultural diplomacy was implemented during the context of my research. 

By combining the information present in the aforementioned texts, I will construct an analogous theory of cultural diplomacy’s use in sport. As the texts are all historical accounts of cultural diplomacy’s use in the 20th-century, the variety of perspectives and time periods allow me to have a well-rounded understanding of cultural diplomacy. The texts all agreed that state actors used cultural diplomacy to promote their values to their ideological opponents while projecting a sense of cultural superiority. However, only David-Fox is explicit in acknowledging cultural diplomacy’s effect on domestic populations (\citeyear{david-fox2012}, p. 122). This insight suggests that cultural diplomacy had a multi-directional effect: it influenced audiences inside and outside of the territorial bounds of the state. To explain why the fine arts were the preferred vehicle for cultural diplomacy, Dance for Export and Music, Art and Diplomacy both argue the fine arts could transcend linguistic barriers and political ideologies, therefore increasing its effectiveness in influencing foreign audiences (\cite{prevots2012}, p. 19; \cite{gonçalves2016}, p. 142). Additionally, Mikkonen \& Suutari claim that cultural diplomacy is more effective than traditional diplomacy because of its ability to appeal to emotions. Since we usually associate the fine arts with high culture, all three texts subsequently have an implicit focus on the higher classes of society. Sport is not typically associated with high culture and enjoys support from a wide cross-section of society. Like the fine arts, sport can elicit emotional responses from spectators. As such, studying football allows for an exploration of cultural diplomacy in relation to mass culture. 
