\chapter{Literature review\label{cha:litreivew}}

This thesis focuses on how the German Democratic Republic’s (GDR) cultural diplomatic efforts through football were reflected in the state-run newspaper outlet \textit{Neues Deutschland}. To comprehensively engage with the research question, I have surveyed publications that examine the topics of culture, propaganda and ideology, along with additional literature on cultural, public and sport diplomacy more broadly to provide context on the function of cultural diplomacy in relations between and within states. I have also reviewed publications that specifically focus on football in the GDR and the media representation of football in order to demonstrate how football was used as a cultural diplomatic tool. Combined, these works allow me to position this thesis in the intersection between several debates surrounding sport and cultural diplomacy. The interdisciplinary nature of this research enables me to make a scholarly contribution that argues for sport’s inclusion in research on cultural diplomacy.

\section{Culture and diplomacy}

To help elucidate how cultural diplomacy is understood within this literature review, the definitions of culture and diplomacy as outlined in the introduction are repeated here. Culture will be defined as a system of values, customs and ideas shared amongst a group that distinguishes itself from non-members through the use of boundaries (\cite{cohen1995}). Subsequently, nations and states are then (large) forms of social organisation in which common cultures can exist. A state does not exist in isolation. Today’s globalised world encompasses numerous states, where diplomacy is fundamental in maintaining stable interstate relations. Paul Sharp (\citeyear{sharp2003}) defines diplomacy as the fostering and maintaining of relations between separate groups, which enables the peaceful co-existence of groups and the avoidance of unnecessary conflict. Within the context of this thesis, states are the referent objects in my examination of diplomacy between groups.

\section{Between cultural, public and sports diplomacy}

In recent decades, cultural diplomacy has become an integral component in a state’s foreign policy arsenal. As opposed to using the ‘stick’ (usually hard-power strategies that include the use of military force or economic sanctions), states can use the ‘carrot’ of cultural diplomacy - a soft-power approach that influences a foreign audience through the dissemination of culture, values and ideas (\cite{lenczowski2009}, p. 76). Many scholars attribute cultural diplomacy’s increased usage due to the proliferation of transnational connections in today’s globalised world (\cite{snow2008}; \cite{ang2015}; \cite{chitty2016}; \cite{hartig2016}). While one of cultural diplomacy’s primary goals is the avoidance of conflict (as is the general goal of diplomacy), Ien Ang, Yudhishthir Isar \& Phillip Mar expanded the discourse on cultural diplomacy by identifying two other objectives its practice may achieve:  1) the advancement of national interests through the positive presentation of the nation to the world; and 2) the promotion of a harmonious international order for the benefit of all (\citeyear{ang2015}, p. 370). As such, cultural diplomacy has become an attractive foreign policy alternative for interstate relations due its non-military approach. Depending on the foreign policy objectives of the state, it can provide opportunities to either project a sense of a state’s cultural superiority over another or foster mutual understanding between states.

Since culture and diplomacy are themselves contentious terms, this thesis is a contribution addressing the lack of clarity in how cultural diplomacy is discussed in scholarly debates. Within parts of the International Relations discipline, cultural diplomacy has been conflated with another soft-power approach: public diplomacy. However, there are some distinctions between the two types of diplomacy. Whereas cultural diplomacy is a governmental practice that uses official agents and envoys to enact foreign policy objectives through activities such as cultural exchange programmes (\cite{ang2015}, p. 367), public diplomacy relies on the interaction between private groups and interests from one state with another (\cite{cull2008b}, p. 19). While both types of diplomacy attempt to foster mutual understanding between peoples from different states (\cite{hartig2016}, p. 261), public diplomacy does not explicitly outwardly project the cultural superiority of the state. Rather, it seeks to ameliorate the negative external perceptions associated with a state (\cite{melissen2011}, p. 14). The question then must be asked: if private groups or interests are funded by the state and act in line with the foreign policy objectives of the state, can they then be conceived of as official agents of the state? With the emergence of non-governmental actors (e.g. non-governmental organisations, transnational corporations and even international sports federations) within international relations, it appears that the distinction between what is traditional cultural diplomacy and public diplomacy has become unclear. In the context of the research topic, the GDR footballers were not official ambassadors or envoys of the state. However, since the GDR state sought to control all aspects of society (\cite{fullbrook1995}; \cite{dennis2000}), categorising the footballers as private groups or interests separate from the state becomes problematic. Though the footballer’s activities can be recognised to occupy a liminal space between cultural diplomacy and public diplomacy, the intent of the state’s diplomacy efforts can help clarify whether they were associated with a cultural or public diplomatic approach. Therefore, studying how these footballers were represented in media publications can help delineate cultural diplomacy from public diplomacy.

As this thesis argues that cultural diplomacy can use sport as its vehicle, consideration of what discourse surrounds sports diplomacy is necessary. Today’s globalised world has facilitated the emergence of an extensive and complex international web of sports and sporting organisations which Wolfram Manzenreiter labels the ‘sportscape’ (\citeyear{manzenreiter2008}, p. 414). Much of the existing literature on sports diplomacy focuses on the politics within local and international sporting organisations (\cite{holt1999}; \cite{tomlinson2016}; \cite{cooley2018}), the politicisation of global sporting events such as the FIFA World Cup and Olympics (\cite{xu2008}; \cite{dowse2018}) or the use of sport to foster interstate dialogue (\cite{rowe2018}; \cite{shuman2018}). Stuart Murray’s \textit{Sports Diplomacy} (\citeyear{murray2018}) presents his readers with a comprehensive anthology on sports diplomacy, providing theoretical perspectives of sports role in diplomacy, its history and selected case studies. Much like the aforementioned texts, Murray’s work centres on the political aspects of sports diplomacy. However, there is a brief section that explores how sports diplomacy is used as an expression of a nation’s culture (\citeyear{murray2018}, pp. 97-102). Murray’s insight into how a nation’s cultural system could be embedded into sport for diplomatic purposes opens the door for the research question that is the focus of this thesis. While cultural diplomacy and public diplomacy tend to easily coalesce with each other, it appears that cultural diplomacy and sports diplomacy have so far acted like oil and water. When conceiving of the parts within the cultural system of a nation or state, many publications point toward the somewhat obvious fields of language, music and art (\cite{hanna1987}; \cite{blackingetal1995}; \cite{clifford1998}). Consequently, publications examining cultural diplomacy exclusively focus on music (\cite{fosler-lussier2015}; \cite{mikkonensuutari2016}; \cite{saito2020}), art (\cite{barnhisel2015}; \cite{mikkonensuutari2016}) and dance (\cite{prevots1998}; \cite{kodat2014}; \cite{mcdaniel2014}), subsequently ignoring sport’s potential to provide a research subject on culture. Literature on sports diplomacy is derived from the International Relations discipline, hence, they have a focus on the political and organisational aspects of sport and neglect its cultural characteristic. Therefore, through this interdisciplinary study, I will show that sport can and should be considered as a cultural object within research on cultural diplomacy.

\section{Cultural diplomacy, propaganda and ideology}

To properly engage with how the GDR used \textit{Neues Deutschland} to reflect their cultural diplomacy efforts in football, we need to consider the points at which cultural diplomacy is differentiated from propaganda and ideology recognising that, as both concepts and practices, there may be some overlap between them. As the primary focus of cultural diplomacy is on influencing a foreign audience through the dissemination of culture, scholars have occasionally used the alternate term of ‘cultural propaganda’ to describe cultural diplomacy’s propaganda potential (\cite{prevots1998}; \cite{david-fox2011}; \cite{faircloughwiggins2016}). Making the distinction between the two terms no clearer, Michael David-Fox argues that there has always been a ‘great deal of overlap between propaganda and cultural diplomacy’ (\citeyear{david-fox2011}, p. 15). Though there is debate on whether propaganda is manipulative or neutral (\cite{diggs-brown2011}, p. 48), there is a general consensus in characterising it as an act of persuasion in order to influence people and behaviour (\cite{blackroberts2011}; \cite{auerbachcastronovo2013}; \cite{milleretal2016}). Propaganda’s connection with cultural diplomacy is not accidental. Nicholas Cull recognised the link between the terms when he associated cultural diplomacy’s coinage with the need for a new euphemistic term, which characterised states' soft-power diplomatic programmes (\citeyear{cull2008a}a, p. 19; \citeyear{cull2008b}b, p. 259). While cultural diplomacy may share etymological ties with propaganda, the former is better equipped for the thesis due to its narrower focus on interstate relations.

As the GDR employed a brand of Marxist-Leninist communism to govern its citizens (\cite{grixcooke2002}, p. 17), this thesis will need to consider the topic of ideology. Mary Fullbrook maintained that the East German state was not clearly separate from society as ‘there was to be no area of society uncontrolled by the state’ (\citeyear{fullbrook1995}, p. 19). Shifting the perspective slightly, Anselma Gallinat reformulated Fullbrook’s claim in suggesting there was no separation between official ideology and grassroots practice (\citeyear{gallinat2005}, p. 291). The combination of the two perspectives allows us to view the state and ideology as one in the same. Daniele Conversi has defined ideology as ‘a set of ideas articulated around a socio-political programme’ (\citeyear{conversi2010}, p. 26). In the GDR context, one could recognise Marxism-Leninism as the ideological model that directed the nation. If there was no separation between state and society, ideology and the grassroots, then the cultural values associated with the state’s cultural diplomacy efforts would be the state-endorsed ideology. Cultural diplomacy’s intended target is foreign audiences, however, when the GDR citizens viewed the cultural diplomatic efforts of the state through the reading of \textit{Neues Deutschland}, then it could be seen as the proliferation of ideology and the ‘sustaining of relations of dominance’ (\cite{thompson1990}, p. 58). While cultural diplomacy may not be intended for consumption within the boundaries of the instigating state, the GDR’s context transforms cultural diplomatic efforts into the proliferation of ideology when their citizens are exposed to its dissemination.

\section{Cultural diplomacy through the arts}

As most literature on cultural diplomacy focuses on the arts and high culture, the insights they offer are valuable for a comparable study of sport’s role in cultural diplomacy insofar as both the fine arts and sport are parts of the cultural system. Since existing literature focuses on the arts and high culture, it does not provide a nuanced insight into mass culture, of which sport is a recognisable artefact. Moreover, literature on cultural diplomacy through the fine arts typically uses the context of Soviet-American relations in the 20th-century as a background for analysis. Thus, there are limits to its applicability to the context of this thesis considering that there is a specific focus on the relations between two distinct states who have a shared German heritage.

David-Fox’s \textit{Showcasing the Great Experiment} (\citeyear{david-fox2011}) provides a historical account of how the Soviet Union used the ‘All-Union Society for Cultural Ties Abroad’ (better known by its Russian acronym “VOKS”) to showcase Soviet artistic excellence to a foreign audience. By examining Soviet archives from the interwar period, David-Fox offers an insight into the Soviet state’s perspective on cultural diplomacy and the role it played in projecting the perception of socialism’s superiority. Of particular relevance to this thesis is David-Fox’s insight into how the Soviet Union attempted to improve their international reputation despite being considered culturally ‘backward’ and economically weak (\cite{david-fox2011}, p. 9). Edited by Simo Mikkonen \& Pekka Suutari, \textit{Music, Art and Diplomacy} (\citeyear{mikkonensuutari2016}) is a volume of assorted chapters offering perspectives on cultural diplomacy from both sides of the Iron Curtain. Focusing on the period between the 1940s and 1960s, the volume’s chapters detail specific acts of cultural diplomacy, where states sent artists to their ideological enemy to showcase their talents. Rather than investigating the state’s perception of cultural diplomacy, \textit{Music, Art and Diplomacy} is focused on the perspective from the artists and audience themselves, which offers an alternative to the grand narratives that usually dominate East-West discourse in Cold War research. The volume utilises archival sources alongside oral history through personal testimony to construct narratives on personal experiences in cultural diplomacy. Thus, it provides this thesis with an account from the targets of cultural diplomacy and the impression the artists made. Naima Prevots’ \textit{Dance for Export} (\citeyear{prevots1998}) references sources from official American government archives to show an entirely American viewpoint on cultural diplomacy’s value in the Cold War. \textit{Dance for Export} chronicles how the United States government channelled their cultural diplomacy efforts through the American National Theatre and Academy (ANTA) by sending dancers abroad to exhibit excellence in Western ballet to counteract the perceptions of an ‘uncultured, superficial, and materialistic’ nation (\cite{prevots1998}, p. 7). Though these texts are specific to art and high culture, they are relevant as they provide a comparative case study of how cultural diplomacy was implemented in the context of my research topic.

By drawing on the outlined approaches that examine how cultural values were embedded into the artistic practices utilised in the cultural diplomatic programmes employed by states, I will construct an analogous theory of cultural diplomacy’s use in sport. As the texts are all historical accounts of cultural diplomacy’s use in the 20th-century, the variety of perspectives and time periods allow me to have a well-rounded understanding of cultural diplomacy. The texts all agree that state actors used cultural diplomacy to promote their values to their ideological opponents while projecting a sense of cultural superiority. However, only David-Fox is explicit in acknowledging cultural diplomacy’s effect on domestic populations (\citeyear{david-fox2011}, p. 122). This insight suggests that cultural diplomacy had a multi-directional effect: it influenced audiences inside and outside of the territorial bounds of the state. To explain why the fine arts were the preferred vehicle for cultural diplomacy, \textit{Dance for Export} and \textit{Music, Art and Diplomacy} both argue the fine arts could transcend linguistic barriers and political ideologies, therefore increasing its effectiveness in influencing foreign audiences (\cite{prevots1998}, p. 19; \cite{gonçalves2016}, p. 142). Additionally, Mikkonen \& Suutari claim that cultural diplomacy is more effective than traditional diplomacy because of its ability to appeal to emotions. Since the fine arts are often associated with high culture, all three texts subsequently have an implicit focus on the privileged classes of society. Sport is not typically associated with high culture and enjoys support from a wide cross-section of society. Like the fine arts, sport can elicit emotional responses from spectators, and as such, studying football allows for an exploration of cultural diplomacy in relation to mass culture.

\section{Football and \textit{Neues Deutschland}}

The sport of football provides a research subject rich with cultural attachments, with media coverage of football allowing for an examination into cultural diplomacy. Viewed through the lens of Anthony Cohen’s work on symbolism (\citeyear{cohen1995}, p. 17), football provided the GDR with a signifier that allowed them to associate the state’s cultural values with their national team players. According to Helmut Hanke, everyday cultural life in the GDR was dominated by devotion to the home and the impact of the media (\citeyear{hanke1990}, p. 179). Only \textit{Reisekader} (people who were loyal to the regime) were allowed to travel abroad to watch the national team play and spectators for international games in the GDR were heavily vetted before they were allocated a ticket (\cite{mcdougall2014}, p. 178). Therefore, articles published in \textit{Neues Deutschland} would have been one of the primary ways in which people engaged with football. Alan McDougall’s \textit{The People’s Game} (\citeyear{mcdougall2014}) is the most comprehensive English-language text on football in the GDR. While it offers a thorough historical account of GDR football, McDougall does not discuss GDR culture or media representation in great detail. A collection of literature exists on football’s association with cultural identity (\cite{archetti1994}; \cite{gibbons2014}; \cite{bocketti2016}) and football’s representation in the media (\cite{boylehaynes2004}, \cite{bridgewater2010}; \cite{miller2011}). Publications that interest themselves with national identity have relevance to this thesis in the way they analyse how national culture can be embedded in football. There are however limits to their applicability to this thesis as the texts do not explicitly question how the cultures of separate nations interact with one another. Literature that focuses on football’s media representation give an insight into how media organisations construct narratives around the sport. These texts are limited in their suitability to this thesis as they lack analyses of cultural representations produced in media publications and instead primarily focus on the economics of football and the sport’s growing ‘brand’ (\cite{bridgewater2010}). Furthermore, I have yet to find any English-language publications examining the influence that \textit{Neues Deustchland} had within the GDR. Although I have achieved German-language proficiency at the C1 level (CEFR standard), this thesis is positioned as an English-language work. Due to the dearth of English-language literature on \textit{Neues Deutschland}, this thesis utilises publications that specifically engage with the research topic in the German language as an exception. \textit{Denver Clan und Neues Deutschland} (\cite{meyen2003}) and \textit{Sattsam bekannte Uniformität} (\cite{meyenschweiger2008}) and are two publications that go beyond a rudimentary examination of \textit{Neues Deutschland’s} influence in the GDR and instead demonstrate how content analysis of articles can be performed. While the two texts could suffer from post-unification distortion (\cite{fullbrook1995}, p. 7), they apply to this thesis in the way they can guide the methodology. As the majority of people in the GDR who engaged with football did so through the consumption of media, the articles published by \textit{Neues Deutschland} provide this thesis with a primary source of how football was represented in the context of the research topic. As such, the thesis aims to analyse the text in \textit{Neues Deutschland} articles relevant to GDR football between November 1973 and June 1974 to uncover how football was used as a cultural diplomatic tool.